\documentclass[a4paper,12pt,titlepage]{article} %page properties
\usepackage[headings]{fullpage} %for fullpage with headings
\usepackage{fancyhdr} %for headings
\usepackage{graphicx} %for label
\usepackage{tabularx} %for fullpage tables
\usepackage{longtable} %for tables on multiple pages
\usepackage{hyperref} %for links
\usepackage{mathtools} %for math symbols

% defining headings
\pagestyle{fancy}
\fancyhead{}
\fancyhead[LE,RO]{PAMI 2015-2016 \hfill FORMULAS SUMMARY}

\begin{document}
	\begin{titlepage}
		
		%defining university
		\begin{center}
			POLITECNICO DI MILANO --- COMO CAMPUS\\
			\vspace{10pt}
			\includegraphics[scale=0.1]{logo-polimi.png}\\
			\vspace{10pt}
			PATTERN ANALYSIS AND MACHINE INTELLIGENCE 2015-2016\\
			prof. Matteo Matteucci	
			\line(2,0){500}
		\end{center}
		
		\vspace{60pt}	
		%defining work type
		\begin{center}
			{\Huge \textbf{Formulas Summary}}\\
		\end{center}
		
		\vspace{60pt}
		
		%defining project repository
		\begin{center}
			{\large Project Repository}
		\end{center}
		\begin{tabularx}{\textwidth}{|X|}
			\hline
			\href{https://github.com/attillax/PAMI-2015}{Click}\\
			\hline
		\end{tabularx}
		
		\vspace{20pt}
		
		%defining team members
		\begin{center}
			{\large Team Members}
		\end{center}
		\begin{tabularx}{\textwidth}{|X|X|X|}
			\hline
			ID & Surname & Name\\
			\hline
			10460625 & Golubeva & Svetlana\\
			\hline
			&  & \\
			\hline
			&  & \\
			\hline
			&  & \\
			\hline
		\end{tabularx}
		
		\vspace{\fill}
		\begin{center}
			\line(2,0){500}
		\end{center}
		
	\end{titlepage}
% % % % % % % % % % % % % % % % % % % % % % % % % % % % % % % % %
\tableofcontents

% % % % % % % % % % % % % % % % % % % % % % % % % % % % % % % % %
\newpage
\section{Statistical learning}
\begin{itemize}
	\item Bias, Variance, irreducible error, expected prediction error
	\item Flexibility, complexity
	\item Bias-variance trade-off (the relations among MSE (test \& trainig), Expected prediction error) (regr)
	\item minimum avg. test error rate (class)
	\item Plots
	\item LDA, decision boundaries
	\item inference vs prediction, \# of observations vs \# of predictors
	\item non-linear functions
	\item Bayes classifier and bayes error rate
	\item Example of a solution for LDA
\end{itemize}

% % % % % % % % % % % % % % % % % % % % % % % % % % % % % % % % %
\newpage
\section{Linear regression }
\begin{itemize}
	\item MSE
	\item Manual computations for linear model ($ \hat{}, \bar{}, \beta$, etc)
	\item training \& test RSS
	\item convenience intervals
	\item null hypothesis
\end{itemize}

% % % % % % % % % % % % % % % % % % % % % % % % % % % % % % % % %
\newpage
\section{Classification}
\begin{itemize}
	\item Discriminative methods
	\item Generative methods
	\item KNN
	\item Euclidean distance
	\item LDA, logistic regression
	\item Manual computations (discriminants, boundary equations, drawings)
	\item QDA (parameters)
	\item the curse of dimensionality
	\item estimation of probabilities
\end{itemize}

% % % % % % % % % % % % % % % % % % % % % % % % % % % % % % % % %
\newpage
\section{Clustering}
\begin{itemize}
	\item SSE, accuracy (internal/external),
	\item K-Means
	\item Hierarchical (agglomerative)
	\item Mixture of Gaussians
	\item DBSCAN
	\item K-medoids
	\item Fuzzy C-means
	\item Jarvis-Patrick
	\item linkage techniques
	\item metrics for the distance between clusters
\end{itemize}

% % % % % % % % % % % % % % % % % % % % % % % % % % % % % % % % %
\end{document}
