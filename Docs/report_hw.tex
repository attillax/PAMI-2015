\documentclass[a4paper,12pt,titlepage]{article} %page properties
\usepackage[headings]{fullpage} %for fullpage with headings
\usepackage{fancyhdr} %for headings
\usepackage{graphicx} %for label
\usepackage{tabularx} %for fullpage tables
\usepackage{longtable} %for tables on multiple pages

\usepackage{hyperref}

% defining headings
\pagestyle{fancy}
\fancyhead{}
\fancyhead[LE,RO]{PAMI 2015-2016 \hfill HOMEWORK REPORT}

\begin{document}
\begin{titlepage}

%defining university
\begin{center}
	POLITECNICO DI MILANO --- COMO CAMPUS\\
	\vspace{10pt}
	\includegraphics[scale=0.1]{logo-polimi.png}\\
	\vspace{10pt}
	PATTERN ANALYSIS AND MACHINE INTELLIGENCE 2015-2016\\
	prof. Matteo Matteucci	
	\line(2,0){500}
\end{center}

\vspace{60pt}

%defining work type
\begin{center}
	{\Huge \textbf{Homework report}}\\
\end{center}

\vspace{60pt}

%defining project repository
\begin{center}
	{\large Project Repository}
\end{center}
\begin{tabularx}{\textwidth}{|X|}
	\hline
	\href{https://github.com/attillax/PAMI-2015}{Click}\\
	\hline
\end{tabularx}

\vspace{20pt}

%defining team members
\begin{center}
	{\large Team Members}
\end{center}
\begin{tabularx}{\textwidth}{|X|X|X|}
	\hline
	ID & Surname & Name\\
	\hline
	10460625 & Golubeva & Svetlana\\
	\hline
	 &  & \\
	\hline
	 &  & \\
	\hline
	 &  & \\
	\hline
\end{tabularx}

\vspace{\fill}
\begin{center}
	\line(2,0){500}
\end{center}

\end{titlepage}

% % % % % % % % % % % % % % % % % % % % % % % % % % % % % % % % %
\tableofcontents

% % % % % % % % % % % % % % % % % % % % % % % % % % % % % % % % %
\newpage
\listoftables

\listoffigures

% % % % % % % % % % % % % % % % % % % % % % % % % % % % % % % % %
\newpage
\section{Preface}
Initial task and sources are available by the following  \href{http://davide.eynard.it/2016/01/11/statistical-learning-with-r-2016-edition/}{link}. \\

The homework consists of four parts:
\begin{itemize}
	\item Linear regression.
	\item The curse of dimensionality.
	\item Classification with LDA.
	\item Color quantization with K-means.
\end{itemize}

% % % % % % % % % % % % % % % % % % % % % % % % % % % % % % % % %
\newpage
\section{Linear regression}
\subsection{Task}
Try to run \underline{LinearRegressionDemo.R}: supposing your current working directory is the one where you unpacked the R files, type:
\begin{verbatim}
source("LinearRegressionDemo.R",print.eval=TRUE)
\end{verbatim}

The print.eval parameter is needed to show you the output of some commands such as \textit{summary} in the context of the \textit{source} command.

Run the demo and try answering the questions you find there. In some cases you should be able to do that immediately after looking at the results, in others you will first need to add few lines of code to actually \textit{get} any result. If you find yourself stuck anywhere, all the material you should need is either in the script itself or in the lab notes.

\subsection{Solution}


% % % % % % % % % % % % % % % % % % % % % % % % % % % % % % % % %
\newpage
\section{The curse of dimensionality}
\subsection{Task}
Try to run \underline{CurseDimDemo.r}: supposing your current working directory is the one where you unpacked the R files, type
\begin{verbatim}
source("CurseDimDemo.r",print.eval=TRUE)
\end{verbatim}

The print.eval parameter is needed to show you the output of some commands such as \textit{summary} in the context of the \textit{source} command.

Run the demo and try answering the questions you find there. In some cases you should be able to do that immediately after looking at the results, in others you will first need to add few lines of code to actually \textit{get} any result. If you find yourself stuck anywhere, all the material you should need is either in the script itself or in the lab notes.

\subsection{Solution}


% % % % % % % % % % % % % % % % % % % % % % % % % % % % % % % % %
\newpage
\section{Classification with LDA}
\subsection{Task}
Try to run \textit{ClassificationDemo.R}: supposing your current working directory is the one where you unpacked the R files, type
\begin{verbatim}
source("ClassificationDemo.R")
\end{verbatim}

Run the demo and try answering the questions you find there. In some cases you should be able to do that immediately after looking at the results, in others you will first need to add few lines of code to actually \textit{get} any result. If you find yourself stuck anywhere, all the material you should need is either in the script itself or in the lab notes.


\subsection{Solution}

% % % % % % % % % % % % % % % % % % % % % % % % % % % % % % % % %
\newpage
\section{Color quantization with K-means}
\subsection{Task}
Try to run \textit{ClusteringDemo.R}: supposing your current working directory is the one where you unpacked the R files, type
\begin{verbatim}
source("ClusteringDemo.R")
\end{verbatim}

Run the demo and try answering the questions you find there. In some cases you should be able to do that immediately after looking at the results, in others you will first need to add few lines of code to actually \textit{get} any result. If you find yourself stuck anywhere, all the material you should need is either in the script itself or in the lab notes.


\subsection{Solution}

% % % % % % % % % % % % % % % % % % % % % % % % % % % % % % % % %
\newpage
\section{Technical Notes}

\subsection{Chronology}
\begin{longtable}{|c|p{13cm}|}
	\hline
	Date & Content\\
	\hline
	20-01-2016 & Release\\
	\hline
	20-01-2016 & Add to repoitory\\
	\hline
	20-01-2016 & Solution\\
	\hline
	& Submission\\
	\hline
	& \\
	\hline
	& \\
	\hline
	& \\
	\hline
	\caption{project's chronology.}
\end{longtable}

\subsection{List of tools}
\begin{table}[h]
	\begin{tabularx}{\textwidth}{|l|X|}
		\hline
		What & Which \\
		\hline
		OS & Linux \\
		\hline
		Lang & R, \TeX \\
		\hline
		IDE & R-Studio, \TeX-Studio, Emacs\\
		\hline
		& Terminal\\
		\hline
		& \\
		\hline
	\end{tabularx}
	\caption{list of tools.}
\end{table}

% % % % % % % % % % % % % % % % % % % % % % % % % % % % % % % % %
\end{document}
